\documentclass[FM,DP]{tulthesis}
% tento dokument používá balíky specifické pro XeLaTeX a lze jej přeložit
% jen XeLaTeXem, nemáte-li instalována použitá (komerční) písma, změňte
% nebo vymažte příkazy \set...font na následujících řádcích

\newcommand{\verze}{1.3}

\usepackage{polyglossia}
\setdefaultlanguage{czech}

% fonty
\usepackage{fontspec}
\usepackage{xunicode}
\usepackage{xltxtra}
\setmainfont[Mapping=tex-text,BoldFont={* Bold},Numbers=OldStyle]{Baskerville 10 Pro}
\setsansfont[Mapping=tex-text,BoldFont={* Bold},Numbers=OldStyle]{Myriad Pro}
\setmonofont[Scale=MatchLowercase]{Vida Mono 32 Pro}

% příkazy specifické pro tento dokument
\newcommand{\argument}[1]{{\ttfamily\color{\tulcolor}#1}}
\newcommand{\prikaz}[1]{\argument{\textbackslash #1}}
\newenvironment{myquote}{\begin{list}{}{\setlength\leftmargin\parindent}\item[]}{\end{list}}
\newenvironment{listing}{\begin{myquote}\color{\tulcolor}}{\end{myquote}}
\sloppy

% deklarace pro titulní stránku
\TULtitle{Třída tulthesis pro \LaTeX\ verze~\verze}{tulthesis \LaTeX\ class version~\verze}
\TULprogramme{N2612}{Elektrotechnika a informatika}{Electrotechnology and informatics}
\TULbranch{1802T007}{Informační technologie}{Information technology}
\TULauthor{doc. RNDr. Pavel Satrapa, Ph.D.}
\TULsupervisor{doc. RNDr. Pavel Satrapa, Ph.D.}
\TULyear{2016}

\begin{document}

\ThesisStart{male}

\begin{abstractCZ}
Tato zpráva popisuje třídu \texttt{tulthesis} pro sazbu absolventských prací
Technické univerzity v~Liberci pomocí typografického systému \LaTeX.
\end{abstractCZ}

\vspace{2cm}

\begin{abstractEN}
This report describes the \texttt{tulthesis} package for Technical university of
Liberec thesis typesetting using the \LaTeX\ typographic system.
\end{abstractEN}

\clearpage

\begin{acknowledgement}
Rád bych poděkoval všem, kteří přispěli ke vzniku tohoto dílka.
\end{acknowledgement}

\tableofcontents

\clearpage

\begin{abbrList}
\textbf{TUL} & Technická univerzita v~Liberci \\
\textbf{FM} & Fakulta mechatroniky, informatiky a mezioborových studií
Technické univerzity v~Liberci
\end{abbrList}

\chapter{Základy použití}

Návody tohoto typu poměrně často obsahují úvod do \LaTeX u jako takového. Já se
s~dovolením omezím jen na popis samotné třídy \argument{tulthesis} a budu
předpokládat, že jste v~otázkách \LaTeX u dostatečně zorientováni. Pokud
to není pravda, snad vám pomůže můj \emph{\LaTeX\ pro pragmatiky}, který
najdete na adrese

\begin{myquote}
\href{http://www.nti.tul.cz/~satrapa/docs/latex/}{\emph{http://www.nti.tul.cz/\textasciitilde satrapa/docs/latex/}}
\end{myquote}

Třída \argument{tulthesis} pro typografický systém \LaTeX\ slouží pro sazbu
absolventských prací Technické univerzity v~Liberci. Jedná se o~modifikaci
standardní třídy \argument{report}, která byla rozšířena o~konstrukce pro snadnou
sazbu titulní stránky a dalších obvyklých součástí absolventských prací.

Základní použití je velmi jednoduché~-- v~úvodním příkazu
\prikaz{documentclass} uvedete jako jméno třídy hodnotu \argument{tulthesis}.
K~argumentům se ještě dostaneme, tím nejvýznamnějším je samozřejmě fakulta, na
níž práce vzniká. Implicitně se sází titulní strana pro bakalářskou práci.
Jedná-li se o~práci diplomovou, přidejte argument \argument{DP}, pro zprávu
ročníkového projektu \argument{RP}, pro magisterský projekt \argument{MP} a pro
disertační práci \argument{Dis}. Bakalářská práce Fakulty mechatroniky tedy
bude začínat

\begin{listing}
\prikaz{documentclass[FM]\{tulthesis\}}
\end{listing}

zatímco diplomová práce Fakulty strojní

\begin{listing}
\prikaz{documentclass[FS,DP]\{tulthesis\}}
\end{listing}

V~preambuli pak musíte v~deklaracích uvést popisné informace: jméno práce, kód
a název studijního programu a oboru, své jméno a jméno vedoucího. Všechny
předchozí údaje jsou povinné. Kromě nich můžete uvést i rok vzniku práce. Pokud
jej vynecháte implicitně se doplní rok aktuální v~okamžiku překladu. Dalším
nepovinným údajem je identifikační kód práce, pokud je práci přidělen. Slouží
k~tomu příkazy:

\begin{listing}
\prikaz{TULtitle\{\emph{název práce česky}\}\{\emph{anglicky}\}}\\
\prikaz{TULprogramme\{\emph{kód studijního programu}\}\{\emph{název česky}\}\{\emph{anglicky}\}}\\
\prikaz{TULbranch\{\emph{kód studijního oboru}\}\{\emph{název česky}\}\{\emph{anglicky}\}}\\
\prikaz{TULauthor\{\emph{jméno autora}\}}\\
\prikaz{TULsupervisor\{\emph{jméno vedoucího}\}}\\
\prikaz{TULyear\{\emph{rok}\}}\\
\prikaz{TULid\{\emph{identifikační kód}\}}
\end{listing}

Po nich už může přijít \prikaz{begin\{document\}} a bezprostředně za ním příkaz
\prikaz{ThesisStart\{\emph{rod}\}}, kde \argument{\emph{rod}} má hodnotu
\argument{female}, pokud je autorkou studentka, nebo \argument{male}, jestliže
práci píše student. Na hodnotě tohoto povinného parametru závisí rod použitý
v~prohlášení o~autorství. Příkaz \prikaz{ThesisStart} vysází

\begin{itemize}
\item českou a anglickou titulní stránku,
\item prázdnou stránku, která se nahradí zadáním,
\item prohlášení o~autorství a autorských právech.
\end{itemize}

Následuje typicky abstrakt ve dvou jazycích, pro který jsou k~dispozici
prostředí \argument{abstractCZ} a \argument{abstractEN}, poděkování (prostředí
\argument{acknowledgement}), obsah (standardní příkaz
\prikaz{tableofcontents}), seznam zkratek (prostředí \argument{abbrList})
a pak již vlastní text práce. Jelikož vychází z~třídy \argument{report}, je
základním příkazem pro jeho strukturování \prikaz{chapter}.

Typické zahájení diplomové práce na Fakultě mechatroniky bude proto vypadat
nějak takto:

\begin{listing}
\begin{verbatim}
\documentclass[FM,DP]{tulthesis}
\usepackage[czech]{babel}
\usepackage[utf8]{inputenc}

\TULtitle{Nějaká práce}{Some thesis}
\TULprogramme{N2612}{Elektrotechnika a informatika}%
{Electrotechnology and informatics}
\TULbranch{1802T007}{Informační technologie}%
{Information technology}
\TULauthor{Bc. Jiří Chytrý}
\TULsupervisor{prof. Ing. Svatopluk Moudrý, CSc.}
\TULyear{2014}

\begin{document}

\ThesisStart{male}

\begin{abstractCZ}
Český abstrakt
\end{abstractCZ}

\vspace{2cm}

\begin{abstractEN}
English abstract
\end{abstractEN}

\begin{acknowledgement}
Vám poděkování a lásku vám.
\end{acknowledgement}

\tableofcontents

\clearpage

\begin{abbrList}
\textbf{EU} & Evropská unie \\
\end{abbrList}

\chapter{Úvod}

Ve své diplomové práci...
\end{verbatim}
\end{listing}

Kompletní příklad použití třídy najdete ve zdrojovém textu tohoto dokumentu,
který je součástí distribuce vizuálního stylu TUL pro \LaTeX.


\chapter{Závislosti a instalace}

Třída samotná využívá pouze standardní součásti \LaTeX u: třídu
\argument{report} a balíky \argument{ifthen}, \argument{tabularx} a
\argument{tul}. Její závislosti jsou tedy prakticky nulové. Další závislosti
ovšem plynou z~použití balíku \argument{tul}, který vyžaduje několik dalších.
Podrobnosti najdete v~jeho dokumentaci.

Třída je distribuována v~jednom archivu společně s~balíkem \argument{tul} a
nemá proto samostatnou instalaci. Sama je tvořena jen souborem
\emph{tulthesis.cls}, který stačí umístit kamkoli, kde jej vaše instalace
\LaTeX u najde. 

Oficiální distribuční adresou je

\begin{listing}
\href{http://www.nti.tul.cz/~satrapa/vyuka/latex-tul/}{\emph{http://www.nti.tul.cz/\textasciitilde satrapa/vyuka/latex-tul/}}
\end{listing}


\chapter{Volby}

Činnost balíku lze ovlivňovat různými volbami. Ty lze rozdělit do několika
skupin:

\begin{description}
\item[Vlastní volby] určují typ práce:

\begin{itemize}
\item \argument{DP}~-- sází se diplomová práce.
\item \argument{BP}~-- sází se bakalářská práce (ve skutečnosti nedělá nic,
protože sazba bakalářské práce je implicitní hodnotou, je zařazena jen pro
zachování symetrie voleb).
\item \argument{Dis}~-- sází se disertační práce.
\item \argument{RP}~-- sází se ročníkový (bakalářský) projekt.
\item \argument{MP}~-- sází se magisterský projekt.
\end{itemize}

\item[Volby balíku \argument{tul}] jsou předány tomuto balíku. Zde uvedu jen
ty, které mají na vzhled práce přímý dopad:

\begin{itemize}
\item \argument{FS},
\argument{FT},
\argument{FP},
\argument{EF},
\argument{FA},
\argument{FM},
\argument{FZS} a
\argument{CXI}~-- určují fakultu, na které práce vzniká; mají dopad jednak na
titulní stránku, jednak na volbu fakultní barvy.

\item \argument{bw}~-- sází se černobíle, titulní stránka a nadpisy částí textu
nebudou používat barvy. Lze předpokládat, že tato volba bude u~absolventských
prací dost obvyklá.
\end{itemize}

\item[Volby třídy \argument{report}] se předají této třídě a mají
standardní dopad. V~základu se třída volá s~volbami \argument{a4paper} a
\argument{12pt}, protože práce se standardně sázejí dvanáctibodovým písmem na
papír formátu A4. Tyto dvě volby jsou pevné, případnými dalšími můžete ovlivnit
chování třídy, například nastavit oboustranný tisk nebo způsob zarovnání
matematických vzorců.

\end{description}


\chapter{Příkazy a prostředí}

V~této kapitole najdete přehled příkazů a prostředí poskytovaných třídou
\argument{tulthesis}. Nejsou řazeny abecedně (na to je jich příliš málo), ale
v~pořadí, které odpovídá pořadí jejich použití v~dokumentu.


\section{Příkazy pro preambuli}

Cílem této skupiny příkazů je definovat důležité popisné informace o~práci.

\begin{description}

\item[\prikaz{TULtitle\{\emph{název práce česky}\}\{\emph{anglicky}\}}]
definuje název práce. První argument obsahuje jeho českou verzi, druhý verzi
anglickou. Povinný příkaz.

\item[\prikaz{TULprogramme\{\emph{kód studijního programu}\}\{\emph{název
programu česky}\}\{\emph{anglicky}\}}] definuje kód a název studijního
programu, v~jehož rámci je práce vytvořena. Kód je pro obě jazykové verze
shodný, proto se zadává samostatným parametrem. Povinný příkaz.

\item[\prikaz{TULbranch\{\emph{kód studijního oboru}\}\{\emph{název
oboru česky}\}\{\emph{anglicky}\}}] definuje kód a název studijního
oboru, v~jehož rámci je práce vytvořena. Povinný příkaz. Může se vyskytnout
vícekrát~-- pokud student studuje kombinaci několika oborů (typické pro FP),
uvede po jednom příkazu \prikaz{TULbranch} pro každý obor.

\item[\prikaz{TULauthor\{\emph{jméno autora}\}}] definuje titul a jméno autora
práce. Povinný příkaz.

\item[\prikaz{TULsupervisor\{\emph{jméno vedoucího}\}}] definuje titul a jméno
vedoucího práce. Povinný příkaz.

\item[\prikaz{TULyear\{\emph{rok vytvoření práce}\}}] definuje rok, ve kterém
práce vznikla. Příkaz je nepovinný, pokud jej vynecháte, doplní se při překladu
automaticky aktuální rok. Přesto doporučuji jej použít, jinak by při případném
pozdějším překladu mohla vzniknout iluze, že práce vznikla v~pozdějším roce,
než ve skutečnosti.

\item[\prikaz{TULid\{\emph{identifikační kód}\}}] určuje identifikační kód,
který některé fakulty absolventským pracím přidělují. Nepovinný příkaz, pokud
práce nemá přidělen kód, jednoduše jej vynechte.

\item[\prikaz{TULthesisType\{\emph{typ práce česky}\}\{\emph{anglicky}\}}]
umožňuje definovat jiný typ práce, než na které pamatují volby balíku
\argument{BP}, \argument{DP} a další. V~prvním argumentu uveďte český
název typu práce, ve druhém jeho anglický ekvivalent. Například pro výzkumnou
zprávu by preambule obsahovala

\begin{listing}
\begin{verbatim}
\TULthesisType{Výzkumná zpráva}{Research report}
\end{verbatim}
\end{listing}

\end{description}


\section{Příkazy a prostředí pro tělo práce}

Tyto příkazy a prostředí generují viditelný výstup a generují jednotlivé
součásti, především pro úvodní partie práce.

\begin{description}

\item[\prikaz{ThesisStart\{\emph{rod}\}}] vygeneruje standardní zahájení
práce~-- titulní list, list pro nahrazení zadáním a prohlášení o~autorství a
autorských právech. Parametr \argument{\emph{rod}} má hodnotu
\argument{female} nebo \argument{male} a určuje, zda bude prohlášení uvedeno
v~rodě ženském nebo mužském. Ve skutečnosti se jedná o~zkratku volající
následující trojici příkazů.

\item[\prikaz{ThesisTitle\{\emph{jazyk}\}}] vygeneruje titulní list práce.
Informace čerpá z~výše uvedených deklaračních příkazů \prikaz{TULtitle} a spol.
Argument určuje, jaká jazyková verze se má vysázet. K dispozici je česká
(hodnota \argument{CZ}) a anglická (\argument{EN}) verze titulního listu. Pokud
použijete \prikaz{ThesisStart}, nemusí vás zajímat~-- je určen pro situace, kdy
vám standardní zahájení práce nevyhovuje a chcete je nějakým způsobem upravit.

\item[\prikaz{Assignment}] vygeneruje prázdný list, který se v~práci nahradí
originálem jejího zadání. Má dva účely: upomíná, abyste nezapomněli vložit
zadání, a zajišťuje správné číslování stránek (list se zadáním se počítá mezi
stránky práce, přestože na něm číslo není uvedeno). Pokud použijete
\prikaz{ThesisStart}, nemusí vás zajímat~-- je určen pro situace, kdy vám
standardní zahájení práce nevyhovuje a chcete je nějakým způsobem upravit.

\item[\prikaz{Declaration\{\emph{rod}\}}] vygeneruje prohlášení o~autorství a
autorských právech. Parametr \argument{\emph{rod}} má hodnotu
\argument{female} nebo \argument{male} a určuje, zda bude prohlášení uvedeno
v~rodě ženském nebo mužském. Pokud použijete \prikaz{ThesisStart}, nemusí vás
zajímat~-- je určen pro situace, kdy vám standardní zahájení práce nevyhovuje a
chcete je nějakým způsobem upravit.

\item[\argument{abstractCZ}] je prostředí pro sazbu českého abstraktu práce.
Nedělá mnoho~-- vysází nadpis „Abstrakt“ a zúží řádek, protože abstrakty
obvykle bývají krátké a na plnou šířku řádku nevypadají dobře. Je-li váš
abstrakt dlouhý, použijte nepovinný argument \argument{wide}, který vynechá
zúžení a sází abstrakt na celou šířku řádku:

\begin{listing}
\begin{verbatim}
\begin{abstractCZ}[wide]
Široký text abstraktu.
\end{abstractCZ}
\end{verbatim}
\end{listing}

\item[\argument{abstractEN}] je prostředí pro sazbu anglického abstraktu práce.
Od \argument{abstractCZ} se liší jen anglickým nadpisem.

\item[\argument{acknowledgement}] je prostředí pro poděkování. Opět se chová
stejně~-- vysází nadpis „Poděkování“ a sází svůj obsah do užších řádků, pokud
nepoužijete nepovinný argument \argument{wide}.

\item[\argument{abbrList}] je prostředí pro sazbu seznamu zkratek. Obsah
seznamu zkratek je koncipován jako tabulka se dvěma sloupci. Zkratku od popisu
proto oddělte znakem \argument{\&} a popis zakončete
\argument{\textbackslash\textbackslash}. Použití vypadá nějak takto:

{\color{\tulcolor}
\begin{verbatim}
\begin{abbrList}
HTML & HyperText Markup Language \\
CSS & Cascading Style Sheets \\
\end{abbrList}
\end{verbatim}}

Seznam zkratek je nejproblémovější konstrukcí a pokud jeho rozsah překročí
stránku, nelze prostředí \argument{abbrList} použít. Věnuje se mu jedna
z~poznámek v~následující části.

\end{description}


\chapter{Závěrečné poznámky}

Několik poznámek a povzdechů závěrem.


\section{Kombinace s~dalšími balíky}

Snažil jsem se spíše o~minimalistický přístup, aby třída nedělala víc než musí
a pokud možno se nepletla do cesty různým balíkům. Její činnost, která by
mohla mít dopady na kompatibilitu s~dalšími balíky, se omezuje na:

\begin{itemize}
\item Nastavení velikosti papíru a okrajů. K~tomu se používají standardní
prostředky \LaTeX u~-- volba \argument{a4paper} a rozměry
\prikaz{oddsidemargin} a spol.

\item Nastavení zápatí stránky prostřednictvím balíku \argument{fancyhdr}.
Chcete-li zasahovat do záhlaví nebo zápatí stránek, dělejte to příkazy tohoto
balíku.

\item Změna formátu nadpisu kapitol pomocí balíku \argument{titlesec}. Opět,
pokud vám nevyhovuje, využijte nástroje poskytované tímto balíkem ke změně.

\end{itemize}

Nasazení dalších balíků a nástrojů pro změnu vzhledu dokumentu či rozšíření
možností sazby by nemělo stát nic v~cestě. Nechávám to na vás.

Například jsem rezignoval na nastavení konkrétního písma pro sazbu práce,
protože písmo a způsob jeho volby závisí jak na písmech dostupných ve vašem
systému, tak na konkrétní variantě \LaTeX u, kterou používáte. Sázíte-li
pdf\-\LaTeX\-em, můžete sáhnout po některém ze standardních balíků typu
\argument{palatino} či \argument{newcent}. Pokud používáte \XeLaTeX, máte
k~dispozici balík \argument{fontspec} (příklad použití si můžete prohlédnout ve
zdrojovém textu tohoto dokumentu).


\section{Seznam zkratek}

Upřímně řečeno, pro seznam zkratek jsem nenašel dobré řešení. Asi nejvhodnější
je použít dvousloupcovou tabulku, v~jejímž levém sloupci se nacházejí zkratky a
v~pravém jejich významy. Problém spočívá v~omezenosti tabulkového modelu
v~\LaTeX u, který neumí automaticky přizpůsobit šířku tabulky stránce a neumí
rozdělit tabulku na několik stránek. Pochopitelně existují rozšiřující balíky,
které to dokážou, ale nikoli oboje najednou.

Vzhledem k~tomu, že ve většině prací seznam zkratek nepřekročí rozsah jedné
stránky, dal jsem při definici prostředí \argument{abbrList} přednost variantě
s~jednodušším použitím, ovšem omezené na jednu stránku: použití balíku
\argument{tabularx}, který implementuje tabulky s~adaptivními odstavcovými
sloupci.

Je-li zkratek mnoho, nelze použít připravené prostředí a je třeba vytvořit
seznam vlastními silami. Začněte nadpisem, který by neměl být číslován, ale
rozhodně by měl být zařazen do obsahu, takže

\begin{listing}
\begin{verbatim}
\section*{Seznam zkratek}
\addcontentsline{toc}{section}{Seznam zkratek}
\end{verbatim}
\end{listing}

Za ním následuje vlastní seznam, který lze řešit například jako vícestránkovou
tabulku. V~tom případě přidejte do preambule dokumentu

\begin{listing}
\begin{verbatim}
\usepackage{longtable}
\end{verbatim}
\end{listing}

a vlastní seznam zahajte konstrukcí

\begin{listing}
\begin{verbatim}
\noindent\begin{longtable}[l]{@{}lp{12cm}@{}}
\end{verbatim}
\end{listing}

Tělo seznamu tvoří opět dvousloupcová tabulka, kde se šířka prvního sloupce
určí automaticky podle nejširší zkratky a šířka druhého je pevně dána, zde
12~cm. Podle konkrétního obsahu musíte najít vhodnou hodnotu, aby seznam
zabíral (zhruba) celou šířku řádku. Jestliže žádná zkratka nemá význam delší
než jeden řádek, můžete si ušetřit trápení a konstrukci \argument{p\{12cm\}}
v~definici sloupců výše nahradit druhým \argument{l}. Tělo seznamu zakončete

\begin{listing}
\begin{verbatim}
\end{longtable}
\end{verbatim}
\end{listing}

Jinou alternativou, která by se mohla hodit i u~kratších seznamů, pokud se
délky jednotlivých zkratek výrazně liší, je rezignovat na tabulky a vysázet
seznam zkratek prostřednictvím standardního prostředí \argument{description}.
Tedy něco jako

\begin{listing}
\begin{verbatim}
\section*{Seznam zkratek}
\addcontentsline{toc}{section}{Seznam zkratek}

\begin{description}
\item[HTML] HyperText Markup Language
\item[CSS] Cascading Style Sheets
\end{description}
\end{verbatim}
\end{listing}

Vzhledem k~popsaným problémům jsem zvažoval, zda seznam zkratek do třídy vůbec
přidávat. Nakonec jsem tak udělal především proto, abych zdůraznil, že se jedná
de facto o~povinnou součást absolventských prací.

\end{document}
